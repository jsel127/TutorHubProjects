% Options for packages loaded elsewhere
\PassOptionsToPackage{unicode}{hyperref}
\PassOptionsToPackage{hyphens}{url}
%
\documentclass[
]{article}
\usepackage{amsmath,amssymb}
\usepackage{iftex}
\ifPDFTeX
  \usepackage[T1]{fontenc}
  \usepackage[utf8]{inputenc}
  \usepackage{textcomp} % provide euro and other symbols
\else % if luatex or xetex
  \usepackage{unicode-math} % this also loads fontspec
  \defaultfontfeatures{Scale=MatchLowercase}
  \defaultfontfeatures[\rmfamily]{Ligatures=TeX,Scale=1}
\fi
\usepackage{lmodern}
\ifPDFTeX\else
  % xetex/luatex font selection
\fi
% Use upquote if available, for straight quotes in verbatim environments
\IfFileExists{upquote.sty}{\usepackage{upquote}}{}
\IfFileExists{microtype.sty}{% use microtype if available
  \usepackage[]{microtype}
  \UseMicrotypeSet[protrusion]{basicmath} % disable protrusion for tt fonts
}{}
\makeatletter
\@ifundefined{KOMAClassName}{% if non-KOMA class
  \IfFileExists{parskip.sty}{%
    \usepackage{parskip}
  }{% else
    \setlength{\parindent}{0pt}
    \setlength{\parskip}{6pt plus 2pt minus 1pt}}
}{% if KOMA class
  \KOMAoptions{parskip=half}}
\makeatother
\usepackage{xcolor}
\usepackage[margin=1in]{geometry}
\usepackage{color}
\usepackage{fancyvrb}
\newcommand{\VerbBar}{|}
\newcommand{\VERB}{\Verb[commandchars=\\\{\}]}
\DefineVerbatimEnvironment{Highlighting}{Verbatim}{commandchars=\\\{\}}
% Add ',fontsize=\small' for more characters per line
\usepackage{framed}
\definecolor{shadecolor}{RGB}{248,248,248}
\newenvironment{Shaded}{\begin{snugshade}}{\end{snugshade}}
\newcommand{\AlertTok}[1]{\textcolor[rgb]{0.94,0.16,0.16}{#1}}
\newcommand{\AnnotationTok}[1]{\textcolor[rgb]{0.56,0.35,0.01}{\textbf{\textit{#1}}}}
\newcommand{\AttributeTok}[1]{\textcolor[rgb]{0.13,0.29,0.53}{#1}}
\newcommand{\BaseNTok}[1]{\textcolor[rgb]{0.00,0.00,0.81}{#1}}
\newcommand{\BuiltInTok}[1]{#1}
\newcommand{\CharTok}[1]{\textcolor[rgb]{0.31,0.60,0.02}{#1}}
\newcommand{\CommentTok}[1]{\textcolor[rgb]{0.56,0.35,0.01}{\textit{#1}}}
\newcommand{\CommentVarTok}[1]{\textcolor[rgb]{0.56,0.35,0.01}{\textbf{\textit{#1}}}}
\newcommand{\ConstantTok}[1]{\textcolor[rgb]{0.56,0.35,0.01}{#1}}
\newcommand{\ControlFlowTok}[1]{\textcolor[rgb]{0.13,0.29,0.53}{\textbf{#1}}}
\newcommand{\DataTypeTok}[1]{\textcolor[rgb]{0.13,0.29,0.53}{#1}}
\newcommand{\DecValTok}[1]{\textcolor[rgb]{0.00,0.00,0.81}{#1}}
\newcommand{\DocumentationTok}[1]{\textcolor[rgb]{0.56,0.35,0.01}{\textbf{\textit{#1}}}}
\newcommand{\ErrorTok}[1]{\textcolor[rgb]{0.64,0.00,0.00}{\textbf{#1}}}
\newcommand{\ExtensionTok}[1]{#1}
\newcommand{\FloatTok}[1]{\textcolor[rgb]{0.00,0.00,0.81}{#1}}
\newcommand{\FunctionTok}[1]{\textcolor[rgb]{0.13,0.29,0.53}{\textbf{#1}}}
\newcommand{\ImportTok}[1]{#1}
\newcommand{\InformationTok}[1]{\textcolor[rgb]{0.56,0.35,0.01}{\textbf{\textit{#1}}}}
\newcommand{\KeywordTok}[1]{\textcolor[rgb]{0.13,0.29,0.53}{\textbf{#1}}}
\newcommand{\NormalTok}[1]{#1}
\newcommand{\OperatorTok}[1]{\textcolor[rgb]{0.81,0.36,0.00}{\textbf{#1}}}
\newcommand{\OtherTok}[1]{\textcolor[rgb]{0.56,0.35,0.01}{#1}}
\newcommand{\PreprocessorTok}[1]{\textcolor[rgb]{0.56,0.35,0.01}{\textit{#1}}}
\newcommand{\RegionMarkerTok}[1]{#1}
\newcommand{\SpecialCharTok}[1]{\textcolor[rgb]{0.81,0.36,0.00}{\textbf{#1}}}
\newcommand{\SpecialStringTok}[1]{\textcolor[rgb]{0.31,0.60,0.02}{#1}}
\newcommand{\StringTok}[1]{\textcolor[rgb]{0.31,0.60,0.02}{#1}}
\newcommand{\VariableTok}[1]{\textcolor[rgb]{0.00,0.00,0.00}{#1}}
\newcommand{\VerbatimStringTok}[1]{\textcolor[rgb]{0.31,0.60,0.02}{#1}}
\newcommand{\WarningTok}[1]{\textcolor[rgb]{0.56,0.35,0.01}{\textbf{\textit{#1}}}}
\usepackage{longtable,booktabs,array}
\usepackage{calc} % for calculating minipage widths
% Correct order of tables after \paragraph or \subparagraph
\usepackage{etoolbox}
\makeatletter
\patchcmd\longtable{\par}{\if@noskipsec\mbox{}\fi\par}{}{}
\makeatother
% Allow footnotes in longtable head/foot
\IfFileExists{footnotehyper.sty}{\usepackage{footnotehyper}}{\usepackage{footnote}}
\makesavenoteenv{longtable}
\usepackage{graphicx}
\makeatletter
\def\maxwidth{\ifdim\Gin@nat@width>\linewidth\linewidth\else\Gin@nat@width\fi}
\def\maxheight{\ifdim\Gin@nat@height>\textheight\textheight\else\Gin@nat@height\fi}
\makeatother
% Scale images if necessary, so that they will not overflow the page
% margins by default, and it is still possible to overwrite the defaults
% using explicit options in \includegraphics[width, height, ...]{}
\setkeys{Gin}{width=\maxwidth,height=\maxheight,keepaspectratio}
% Set default figure placement to htbp
\makeatletter
\def\fps@figure{htbp}
\makeatother
\usepackage{soul}
\setlength{\emergencystretch}{3em} % prevent overfull lines
\providecommand{\tightlist}{%
  \setlength{\itemsep}{0pt}\setlength{\parskip}{0pt}}
\setcounter{secnumdepth}{-\maxdimen} % remove section numbering
\ifLuaTeX
  \usepackage{selnolig}  % disable illegal ligatures
\fi
\IfFileExists{bookmark.sty}{\usepackage{bookmark}}{\usepackage{hyperref}}
\IfFileExists{xurl.sty}{\usepackage{xurl}}{} % add URL line breaks if available
\urlstyle{same}
\hypersetup{
  pdftitle={Introduction To R},
  hidelinks,
  pdfcreator={LaTeX via pandoc}}

\title{Introduction To R}
\author{}
\date{\vspace{-2.5em}}

\begin{document}
\maketitle

\hypertarget{establishing-our-basics}{%
\section{\texorpdfstring{\textbf{Establishing our
Basics}}{Establishing our Basics}}\label{establishing-our-basics}}

\textbf{\emph{What is programming?}}

Imagine you are a tutor helping a friend learn a new language. Your
friend, though very smart, doesn't understand English, so you need to
use a special language they do understand. This language is like a
programming language, and your friend is the computer.

\textbf{\emph{What is R?}}

R is a programming language designed for statistical computing and data
analysis.

\textbf{Terms that might show up}

Compiler: converts the stuff you write to stuff machines can understand

\hypertarget{basic-r-syntax}{%
\section{\texorpdfstring{\textbf{Basic R
Syntax}}{Basic R Syntax}}\label{basic-r-syntax}}

\begin{enumerate}
\def\labelenumi{\arabic{enumi}.}
\item
  \ul{\textbf{\emph{Comments}}}: These are used to explain what is going
  on and ignored by the R interpreter. Comments are preceded the `\#'
  symbol

\begin{Shaded}
\begin{Highlighting}[]
\CommentTok{\# This is a comment}
\end{Highlighting}
\end{Shaded}
\item
  \ul{\textbf{\emph{Assignment}}}: You can use the
  `\texttt{\textless{}-}' or `\texttt{=}' symbol for setting variables
  equal to something.

\begin{Shaded}
\begin{Highlighting}[]
\NormalTok{x }\OtherTok{\textless{}{-}} \DecValTok{5} 
\NormalTok{x }\OtherTok{=} \DecValTok{5}
\end{Highlighting}
\end{Shaded}
\item
  \ul{\textbf{\emph{Print Output:}}} The `\texttt{print()}' function is
  used to display the value of a variable.

\begin{Shaded}
\begin{Highlighting}[]
\FunctionTok{print}\NormalTok{(x)}
\end{Highlighting}
\end{Shaded}

\begin{verbatim}
## [1] 5
\end{verbatim}
\item
  \ul{\textbf{\emph{Data Types:}}}

  \begin{enumerate}
  \def\labelenumii{\arabic{enumii}.}
  \item
    \emph{Numeric}: represent real numbers (integers and decimals)
  \item
    \emph{Character}: represent characters or strings (sequence of
    characters)
  \end{enumerate}

\begin{Shaded}
\begin{Highlighting}[]
\CommentTok{\#character variable, you can use either \textquotesingle{} \textquotesingle{} or " " to surround the characters }
\NormalTok{char\_var }\OtherTok{=} \StringTok{\textquotesingle{}Hello World\textquotesingle{}}
\NormalTok{char\_var2 }\OtherTok{=} \StringTok{"Hello World"} 

\FunctionTok{print}\NormalTok{(char\_var)}
\end{Highlighting}
\end{Shaded}

\begin{verbatim}
## [1] "Hello World"
\end{verbatim}

\begin{Shaded}
\begin{Highlighting}[]
\FunctionTok{print}\NormalTok{(char\_var2)}
\end{Highlighting}
\end{Shaded}

\begin{verbatim}
## [1] "Hello World"
\end{verbatim}

  \begin{enumerate}
  \def\labelenumii{\arabic{enumii}.}
  \setcounter{enumii}{2}
  \tightlist
  \item
    \emph{Logical}: represents boolean values (i.e.~`TRUE' or `FALSE')
  \end{enumerate}

\begin{Shaded}
\begin{Highlighting}[]
\NormalTok{boolean\_var }\OtherTok{=} \DecValTok{5} \SpecialCharTok{\textgreater{}} \DecValTok{6}

\FunctionTok{print}\NormalTok{(boolean\_var)}
\end{Highlighting}
\end{Shaded}

\begin{verbatim}
## [1] FALSE
\end{verbatim}

  \begin{enumerate}
  \def\labelenumii{\arabic{enumii}.}
  \setcounter{enumii}{3}
  \tightlist
  \item
    \emph{Vector}: one-dimensional array that can hold element of the
    SAME data type
  \end{enumerate}

\begin{Shaded}
\begin{Highlighting}[]
  \CommentTok{\# Numeric vector}
\NormalTok{  numeric\_vector }\OtherTok{\textless{}{-}} \FunctionTok{c}\NormalTok{(}\DecValTok{1}\NormalTok{, }\DecValTok{2}\NormalTok{, }\DecValTok{3}\NormalTok{, }\DecValTok{4}\NormalTok{, }\DecValTok{5}\NormalTok{)}
  \FunctionTok{print}\NormalTok{(numeric\_vector)}
\end{Highlighting}
\end{Shaded}

\begin{verbatim}
## [1] 1 2 3 4 5
\end{verbatim}

\begin{Shaded}
\begin{Highlighting}[]
  \CommentTok{\# Character vector}
\NormalTok{  char\_vector }\OtherTok{\textless{}{-}} \FunctionTok{c}\NormalTok{(}\StringTok{"apple"}\NormalTok{, }\StringTok{"grapes"}\NormalTok{, }\StringTok{"banana"}\NormalTok{)}

  \FunctionTok{print}\NormalTok{(char\_vector)}
\end{Highlighting}
\end{Shaded}

\begin{verbatim}
## [1] "apple"  "grapes" "banana"
\end{verbatim}

  \begin{enumerate}
  \def\labelenumii{\arabic{enumii}.}
  \setcounter{enumii}{4}
  \tightlist
  \item
    \ul{\textbf{\emph{Other Data Types (You can skip this :) )}}}
  \end{enumerate}

  \begin{itemize}
  \tightlist
  \item
    \emph{Matrix}: a matrix is a 2D array with rows and columns
  \end{itemize}

\begin{Shaded}
\begin{Highlighting}[]
\CommentTok{\# Matrix}
\NormalTok{matrix\_var }\OtherTok{\textless{}{-}} \FunctionTok{matrix}\NormalTok{(}\FunctionTok{c}\NormalTok{(}\DecValTok{1}\NormalTok{, }\DecValTok{2}\NormalTok{, }\DecValTok{3}\NormalTok{, }\DecValTok{4}\NormalTok{, }\DecValTok{5}\NormalTok{, }\DecValTok{6}\NormalTok{), }\AttributeTok{nrow =} \DecValTok{2}\NormalTok{, }\AttributeTok{ncol =} \DecValTok{3}\NormalTok{)}

\FunctionTok{print}\NormalTok{(matrix\_var)}
\end{Highlighting}
\end{Shaded}

\begin{verbatim}
##      [,1] [,2] [,3]
## [1,]    1    3    5
## [2,]    2    4    6
\end{verbatim}

  \begin{itemize}
  \tightlist
  \item
    \emph{Array}: multi-dimensional generalization of a matrix
  \end{itemize}

\begin{Shaded}
\begin{Highlighting}[]
\CommentTok{\# Creating a 3D array }
\NormalTok{array\_var }\OtherTok{\textless{}{-}} \FunctionTok{array}\NormalTok{(}\FunctionTok{c}\NormalTok{(}\DecValTok{1}\NormalTok{, }\DecValTok{2}\NormalTok{, }\DecValTok{3}\NormalTok{, }\DecValTok{4}\NormalTok{, }\DecValTok{5}\NormalTok{, }\DecValTok{6}\NormalTok{), }\AttributeTok{dim =} \FunctionTok{c}\NormalTok{(}\DecValTok{2}\NormalTok{, }\DecValTok{2}\NormalTok{, }\DecValTok{2}\NormalTok{))}

\CommentTok{\# Displaying the array}
\FunctionTok{print}\NormalTok{(array\_var)}
\end{Highlighting}
\end{Shaded}

\begin{verbatim}
## , , 1
## 
##      [,1] [,2]
## [1,]    1    3
## [2,]    2    4
## 
## , , 2
## 
##      [,1] [,2]
## [1,]    5    1
## [2,]    6    2
\end{verbatim}

  \begin{itemize}
  \tightlist
  \item
    \emph{List}: collection of different data types (e.g.~characters,
    vectors, numeric, etc.)
  \end{itemize}

\begin{Shaded}
\begin{Highlighting}[]
\CommentTok{\# List}
\NormalTok{list\_var }\OtherTok{\textless{}{-}} \FunctionTok{list}\NormalTok{(}\AttributeTok{name =} \StringTok{"John"}\NormalTok{, }\AttributeTok{age =} \DecValTok{25}\NormalTok{, }\AttributeTok{is\_student =} \ConstantTok{TRUE}\NormalTok{)}

\FunctionTok{print}\NormalTok{(list\_var)}
\end{Highlighting}
\end{Shaded}

\begin{verbatim}
## $name
## [1] "John"
## 
## $age
## [1] 25
## 
## $is_student
## [1] TRUE
\end{verbatim}

  \begin{itemize}
  \tightlist
  \item
    Data Frame: A data fram is a 2D dimensional table with rows and
    columns where each column can be a different data type
  \end{itemize}

\begin{Shaded}
\begin{Highlighting}[]
\CommentTok{\# Data frame}
\NormalTok{df\_var }\OtherTok{\textless{}{-}} \FunctionTok{data.frame}\NormalTok{(}\AttributeTok{name =} \FunctionTok{c}\NormalTok{(}\StringTok{"Alice"}\NormalTok{, }\StringTok{"Bob"}\NormalTok{, }\StringTok{"Charlie"}\NormalTok{), }\AttributeTok{age =} \FunctionTok{c}\NormalTok{(}\DecValTok{25}\NormalTok{, }\DecValTok{30}\NormalTok{, }\DecValTok{22}\NormalTok{))}

\FunctionTok{print}\NormalTok{(df\_var)}
\end{Highlighting}
\end{Shaded}

\begin{verbatim}
##      name age
## 1   Alice  25
## 2     Bob  30
## 3 Charlie  22
\end{verbatim}

  \begin{enumerate}
  \def\labelenumii{\arabic{enumii}.}
  \setcounter{enumii}{5}
  \tightlist
  \item
    \ul{\textbf{\emph{Indexing}}}: R uses 1-base indexing meaning the
    index of the first element in a vector is 1.
  \end{enumerate}
\end{enumerate}

\begin{Shaded}
\begin{Highlighting}[]
\NormalTok{numeric\_vector[}\DecValTok{1}\NormalTok{]  }\CommentTok{\# Access the first element}
\end{Highlighting}
\end{Shaded}

\begin{verbatim}
## [1] 1
\end{verbatim}

\begin{Shaded}
\begin{Highlighting}[]
\NormalTok{char\_vector[}\DecValTok{2}\NormalTok{]    }\CommentTok{\# Access the second element}
\end{Highlighting}
\end{Shaded}

\begin{verbatim}
## [1] "grapes"
\end{verbatim}

\begin{enumerate}
\def\labelenumi{\arabic{enumi}.}
\setcounter{enumi}{6}
\tightlist
\item
  \ul{\textbf{\emph{Functions}}}: functions are defined using the
  `function()' keyword. The `return()' statement is used to return a
  value from a function
\end{enumerate}

\begin{Shaded}
\begin{Highlighting}[]
\CommentTok{\# Example of a function that adds two numbers together}
\NormalTok{add\_numbers }\OtherTok{\textless{}{-}} \ControlFlowTok{function}\NormalTok{(a, b) \{}
\NormalTok{  result }\OtherTok{\textless{}{-}}\NormalTok{ a }\SpecialCharTok{+}\NormalTok{ b}
  \FunctionTok{return}\NormalTok{(result)}
\NormalTok{\}}

\CommentTok{\# Call the function}
\NormalTok{sum\_result }\OtherTok{\textless{}{-}} \FunctionTok{add\_numbers}\NormalTok{(}\DecValTok{3}\NormalTok{, }\DecValTok{5}\NormalTok{)}
\FunctionTok{print}\NormalTok{(sum\_result)}
\end{Highlighting}
\end{Shaded}

\begin{verbatim}
## [1] 8
\end{verbatim}

\hypertarget{using-r-for-statistics}{%
\section{\texorpdfstring{\textbf{Using R for
Statistics}}{Using R for Statistics}}\label{using-r-for-statistics}}

\hypertarget{descriptive-statistics}{%
\subsection{\texorpdfstring{\textbf{Descriptive
Statistics}}{Descriptive Statistics}}\label{descriptive-statistics}}

\begin{longtable}[]{@{}
  >{\raggedright\arraybackslash}p{(\columnwidth - 4\tabcolsep) * \real{0.3333}}
  >{\raggedright\arraybackslash}p{(\columnwidth - 4\tabcolsep) * \real{0.1852}}
  >{\raggedright\arraybackslash}p{(\columnwidth - 4\tabcolsep) * \real{0.4815}}@{}}
\toprule\noalign{}
\begin{minipage}[b]{\linewidth}\raggedright
Statistics Concept
\end{minipage} & \begin{minipage}[b]{\linewidth}\raggedright
Function in R
\end{minipage} & \begin{minipage}[b]{\linewidth}\raggedright
parameters
\end{minipage} \\
\midrule\noalign{}
\endhead
\bottomrule\noalign{}
\endlastfoot
Mean & mean(x) & x: vector containing data of interest \\
Median & median(x) & x: vector containing data of interest \\
sample standard deviation & sd(x) & x: vector containing data of
interest \\
sample variation & var(x) & x: vector containing data of interest \\
range & range(x) & x: vector containing data of interest \\
Five-number summary & summary(x) & x: vector containing data of
interest \\
\end{longtable}

\textbf{Data Visualization}

\begin{longtable}[]{@{}
  >{\raggedright\arraybackslash}p{(\columnwidth - 4\tabcolsep) * \real{0.2895}}
  >{\raggedright\arraybackslash}p{(\columnwidth - 4\tabcolsep) * \real{0.1974}}
  >{\raggedright\arraybackslash}p{(\columnwidth - 4\tabcolsep) * \real{0.5132}}@{}}
\toprule\noalign{}
\begin{minipage}[b]{\linewidth}\raggedright
Statistics Concept
\end{minipage} & \begin{minipage}[b]{\linewidth}\raggedright
Function in R
\end{minipage} & \begin{minipage}[b]{\linewidth}\raggedright
parameters
\end{minipage} \\
\midrule\noalign{}
\endhead
\bottomrule\noalign{}
\endlastfoot
histogram & hist(x) & x: vector containing data of interest \\
box-and-whisker plot & boxplot(x) & x: vector containing data of
interest \\
& & x: vector containing data of interest \\
\end{longtable}

\textbf{Probability Distribution}

\begin{longtable}[]{@{}
  >{\raggedright\arraybackslash}p{(\columnwidth - 4\tabcolsep) * \real{0.1714}}
  >{\raggedright\arraybackslash}p{(\columnwidth - 4\tabcolsep) * \real{0.3429}}
  >{\raggedright\arraybackslash}p{(\columnwidth - 4\tabcolsep) * \real{0.4857}}@{}}
\toprule\noalign{}
\begin{minipage}[b]{\linewidth}\raggedright
Statistics Concept
\end{minipage} & \begin{minipage}[b]{\linewidth}\raggedright
Function in R
\end{minipage} & \begin{minipage}[b]{\linewidth}\raggedright
Output
\end{minipage} \\
\midrule\noalign{}
\endhead
\bottomrule\noalign{}
\endlastfoot
Normal distribution & pnorm(x, mean = mean, sd = standard deviation) &
probability of a value in the sample being less than or equal to x \\
Binomial Distriubution & dbinom() & probability of a value being equal
to x \\
& pbinom() & \\
Geometric Distribution & dgeom() & \\
& pgeom() & \\
\end{longtable}

\textbf{Hypothesis Testing}

\begin{longtable}[]{@{}
  >{\raggedright\arraybackslash}p{(\columnwidth - 2\tabcolsep) * \real{0.2532}}
  >{\raggedright\arraybackslash}p{(\columnwidth - 2\tabcolsep) * \real{0.7468}}@{}}
\toprule\noalign{}
\begin{minipage}[b]{\linewidth}\raggedright
Statistics Concept
\end{minipage} & \begin{minipage}[b]{\linewidth}\raggedright
Function in R
\end{minipage} \\
\midrule\noalign{}
\endhead
\bottomrule\noalign{}
\endlastfoot
z-test & z.test(data, mu = population mean, sigma = population sd) \\
t-test & t.test(data) \\
\end{longtable}

\hypertarget{other-resources}{%
\section{Other Resources}\label{other-resources}}

\begin{itemize}
\tightlist
\item
  \url{https://www.w3schools.com/r/r_stat_intro.asp}
\end{itemize}

\end{document}
